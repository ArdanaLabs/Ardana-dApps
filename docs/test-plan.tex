\documentclass{article} % Use the custom invoice class (invoice.cls)
\usepackage{todonotes}

\title{dUSD Test Plan}
\begin{document}


\maketitle

\section{Introduction}

\textbf{The purpose of this document is to let the dUSD team decide whether we are ready to deploy.}
It is a living document that serves as a forcing function for discussions around, and agreement on, decisions.

The document starts with the "Interactions" section, a user-centric perspective on what service we would like to offer.
Next come descriptions of requirements we have on those interactions in the "Acceptance criteria" section.
Any important implementation details that will be relevant for building such a system are detailed in the section after that.
Finally, the "Tests" section provide a non-exhaustive list of the way we plan to falsify whether we should deploy.

\section{Interactions}


\subsection{Any User}

\subsubsection{Collect information on the user's vaults}

As a user, I want to be able to collect information on my vaults. By "my
vaults", we mean the vaults created by the user. They can only be used by that
user (except for the liquidation procedure, see below).

\subsubsection{Initialize vault}

As a user, I want to be able to open a new vault.

\subsubsection{Deposit}

As a user, I want to be able to deposit ADA into a vault, so that I may take out
a loan of dUSD (see below).

\subsubsection{Withdrawal}

As a user, I want to be able to withdraw ADA from a vault.

\subsubsection{Take out loan}

As a user, I want to be able to take out a loan in dUSD against the collateral I
put into one of my vaults.

\subsubsection{Pay back loan}

As a user, I want to be able to pay back (part of) a loan in dUSD I created in
one of my vaults. In doing so, I am paying a stability fee to the buffer.

\subsubsection{Liquidate someone else's vault}

...

\subsection{Admin}

\subsubsection{Trigger surplus auction (sick vault)}

\subsubsection{Trigger surplus auction (buy back dana)}

\subsubsection{Trigger debt auction}

\subsubsection{Set liquidation fee}

\subsubsection{Set liquidation ratio}

\subsubsection{Set minimum collateralization ratio}

\subsubsection{Set stability fee}

\subsubsection{Set list of price oracles to consider in OSM}

\subsubsection{Set the time interval the OSM waits before using a price update}

\section{Acceptance criteria}

\subsection{Vaults}

Each user should have access to a website at "www.ardana-vaults.com". This
website must be able to connect to their wallet to find out who they are and
allow them to set up vaults and interact with their existing vaults. \\

Decisions to be made:
\begin{itemize}
  \item Do we want the interface to be a website or an app?
  \item Which domain name will the website live at?
\end{itemize}

\subsubsection{Collect information on the user's vaults}

The vault website looks up the requested information and displays it in an
aesthetic and clear overview. \\

Requested information: How many vaults does the user have, how much collateral
and debt do they each have, what are their collateralization ratios, how
dangerous is it to have that collateralization ratio? \\

Decisions to be made:
\begin{itemize}
  \item Is the list of requested information correct? Anything to be added,
    removed or updated?
\end{itemize}

\subsubsection{Initialize}

The vault website should have a button to open a new vault.

\subsubsection{Deposit}

The vault website has a form that allows each user to pick one of their vaults
and deposit ADA into it. \\

Restrictions:
\begin{itemize}
  \item You have at least the amount of ADA in your wallet that you would like
    to deposit.
\end{itemize}

Decisions to be made:
\begin{itemize}
  \item Do we want to implement a maximum amount of collateral? MakerDAO has
    this, not sure why.
  \item Do we want to implement a mininum amount of collateral? If so, is this
    in total per vault, or per deposit?
\end{itemize}

Questions:
\begin{itemize}
  \item How will we figure out from which UTXO in your wallet to take the ADA?
\end{itemize}

\subsubsection{Withdrawal}

The vault website has a form that allows each user to pick one of their vaults
and withdraw ADA from it. \\

Restrictions:
\begin{itemize}
  \item Withdrawing this amount of ADA from that vault doesn't sink its
    collateralization ratio below the minimum collateralization ratio.
\end{itemize}

\subsubsection{Take out loan}

The vault website has a form that allows a user to pick one of their vaults and
take out a dUSD loan in it, against his collateral. \\

Restrictions:
\begin{itemize}
  \item This transaction doesn't make the collateralization ratio drop below the
    minimum collateralization ratio.
\end{itemize}

\todo{What should the minimum collateralization ratio be?}
\todo{Should the minimum collateralization ratio be different from the
liquidation ratio??}
\todo{Should there be a minimum amount one can take out as a loan? If so, is
this per vault or per loan transaction?}

\subsubsection{Pay back loan}

The vault website has a form that allows a user to pick one of their vaults and
pack back a (part of a) dUSD loan in it. \\

Restrictions:
\begin{itemize}
  \item The amount of dUSD paid back by the user is equal to or smaller than the
    amount of dUSD debt in the vault.
  \item The stability fee for the part of loan that is paid back, is sent to the
    buffer as community profit
\end{itemize}

Design decisions:
\begin{itemize}
  \item What should the minimum collateralization ratio be?
  \item Should the minimum collateralization ratio be different from the
    liquidation ratio??
  \item Should there be a minimum amount one can take out as a loan? If so, is
    this per vault or per loan transaction?
\end{itemize}

\subsubsection{Liquidate someone else's vault}

...

\subsection{Admin}

\subsubsection{Trigger surplus auction (sick vault)}

\subsubsection{Trigger surplus auction (buy back dana)}

\subsubsection{Trigger debt auction}

\subsubsection{Set liquidation fee}

\subsubsection{Set liquidation ratio}

\subsubsection{Set minimum collateralization ratio}

\subsubsection{Set stability fee}

\subsubsection{Set list of price oracles to consider in OSM}

\subsubsection{Set the time interval the OSM waits before using a price update}


\section{Implementation details}

\section{Tests}

% \section{Remarks}
% 
% - Test for Cardano network congestion
% - Test what happens if someone hacks one or more price oracles
% - Other attack vectors which are not specific to our application, should be
%   listed and tested for


\end{document}
