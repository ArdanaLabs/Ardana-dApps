\documentclass{article}
\usepackage[hidelinks]{hyperref}
\usepackage{todonotes}
\setuptodonotes{inline}
\setlength{\parindent}{0em}
\setlength{\parskip}{1pt}

\title{Danaswap specification}
\begin{document}

\maketitle

\tableofcontents
\newpage

\section{Introduction}

This is an early draft of the DanaSwap spec. The document starts with the
``Interactions'' section, defining what the protocol should be able to do. Next
come the ``Acceptance criteria'', containing descriptions of requirements we
have on those interactions. Any important implementation details are detailed in
the ``Implementation Details'' section after that. Finally, the ``Tests''
section provides an exhaustive list of the way we plan to falsify whether we
should deploy.  \\

Note: Every subsection that contain `TBD' (``to be determined'') in the title is
not fully agreed upon yet. Every subsection containing `TBC' (``to be
confirmed'', by Ryan) in the title is mostly agreed upon, but still requires
last confirmation. The other sections are considered law.

\section{Interactions}

\subsection*{Initialize pool with liquidity}

This should create a pool with some initial balance for both its assets, giving
the creator the appropriate number of liquidity tokens. Note that the `admin'
user is determined by a set of three pre-determined keys, which are hardcoded
into the script. Anyone can initialize a pool through the UX. \\

Each pool only contains two assets and has its own liquidity token. Once a pool
is created between assets A and B, the UI no longer allows to create a pool
between these assets.

\subsection*{Swap}

A user can exchange tokens at a rate defined by the constant product invariant,
$x \cdot y = k^2$. The user gets $\Delta y$ in return for paying $\Delta x$,
where $(x + \Delta x) \cdot (y - \Delta y) = k^2$, minus the liquidity fee and
admin fee.

Note: $k$ should not be computed on chain, everything can be done in terms of $k^2$

\subsection*{Add liquidity}

A user should be able to add liquidity to a given pool, and mint that pool’s
liquidity tokens in return. The number of liquidity tokens minted is
proportional to the growth of the pool. Specifically, this means that $\Delta k
/ k = \Delta L / L$, where $k$ is the parameter in the equation $x \cdot y =
k^2$, $L$ is the number of liquidity tokens minted and $\Delta L$ is the number
of liquidity tokens minted again.

Note: Here also $k$ should not be computed on chain, we can use the equivalent
constraint $(\Delta k / k)^2 = (\Delta L / L)^2$ which should avoid the square
root.

\subsection*{Withdraw liquidity}

A user who owns liquidity tokens for a given pool, can burn them in exchange for
getting back a proportional part of the assets left in the pool.

\subsection*{Withdraw liquidity in one asset class}

A user who owns liquidity tokens for a given pool, can choose to withdraw
liquidity in one token, again following the equation $x \cdot y = k^2$. This
endpoint charges the liquidity fee and admin fee.

\subsection*{Kill pool (admin)}

Each pool has a kill switch, which disallows any transaction with the pool
except for the regular ``Withdraw liquidity'' endpoint. Even the ``Withdraw
liquidity in one asset class'' endpoint is no longer allowed. Only the admin
user can kill a pool.

\subsection*{Collect admin fees (admin)}

the admin user can collect the admin fees from a pool which have aggregated from
swaps and single-asset-class withdrawals.

\subsection*{List pools}

The UX show all Danaswap pools in existence.

\subsection*{Show pool info}

For each Danasawp pool in existence, the UX shows the asset amounts in the pool,
total amount of liquidity tokens minted, exchange rates and price of the
liquidity token.

\section{Acceptance criteria}

\begin{itemize}
  \item When initializing a pool, no asset balances can be $0$
  \item Once a pool is initialized between assets $A$ and $B$, the UX disallows
    creating another pool between the same assets
  \item Any user can initialize a pool
  \item The admin user is independent of who created the pool
  \item The admin user can kill pools
  \item Non-admin users cannot kill pools
  \item A pool that is killed, only supports the ``Withdraw liquidity'' endpoint
  \item Small swaps create low slippage
  \item If the balances of the two assets in a pool are roughly equal, their
    exchange rate is roughly $1$
  \item Swaps selling $A$ to the pool in exchange for $B$ increase the exchange
    rate $A \rightarrow B$ and decrease the exchange rate $B \rightarrow A$
  \item When performing a swap, the correct fee is paid
  \item Any user can add liquidity
  \item When adding liquidity, the user receives the correct number of liquidity
    tokens
  \item When withdrawing liquidity, the user receives the correct amount of
    assets
  \item When withdrawing liquidity in one asset class, the correct fee is paid
  \item When creating a new pool and then listing all existing pools, the new
    pool shows up
  \item Each pool is discoverable and traceable, as shown through its ``pool
    info''
  \item When showing a pool's info and then interacting with it (kill, swap, add
    liquidity, withdraw liquidity), the pool's info updates appropriately
  \item Each pool is verifiable, i.e. any pool which has been initialized
    appropriately is considered a legitimate pool and vice versa
\end{itemize}

\section{Implementation details}

This section discusses what we need in order to implement the interactions
mentioned above.

\todo{Write out this section.}

The components we need:
\begin{itemize}
  \item DEX pools
  \item Pool identifying NFTs
  \item Liquidity tokens
\end{itemize}

\subsection*{Initialize pool}

Initialize pool: Initially, $k_0 = x_0 \cdot y_0$ where $x_0$ and $y_0$ are the
initial amounts. The admin user receives $k_0$ liquidity tokens.

\section{Tests}

\todo{Off-chain tests check that what should work on-chain, does work}

\todo{Apropos-Tx???}

\subsection{Unit tests}

The following intended behaviors need to work:
\begin{itemize}
  \item Initializing a protocol and pool works
  \item queries accurately reflect how the pool was setup
  \item valid swaps work
  \item the queries correctly report which swaps would be valid
  \item queries accurately reflect the change from a swap
  \item valid liquidity changes work
  \item queries reflect the liquidity changes
  \item admin can kill a pool
  \item liquidity can be extracted from a dead pool
\end{itemize}

Nick's tests:
\begin{itemize}
  \item When adding liquidity, the number in the pool's datum representing the
    total amount of liquidity minted goes up by the correct amount
\end{itemize}

\subsection{Attacks}

The following attacks need to not work:
\begin{itemize}
  \item Get a pool authorization nft away from a pool (this should include tests at each type of interaction)
  \item non-admin kill a pool
  \item extract too much in an exchange
  \item over mint liquidity
  \item under burn liquidity
  \item burn the wrong pool’s liquidity token
  \item mint the wrong tools liquidity token
\end{itemize}

\end{document}
