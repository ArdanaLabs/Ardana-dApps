\documentclass{article}
\usepackage{hyperref}
\usepackage{todonotes}
\usepackage{graphicx}
\setuptodonotes{inline}
\setlength{\parindent}{0em}
\setlength{\parskip}{1pt}

\title{Danaswap specification}
\begin{document}

\maketitle

\begingroup
  \hypersetup{hidelinks}
  \tableofcontents
\endgroup
\newpage

\section{Introduction}

This is an early draft of the Danaswap spec. The document starts with the
``Interactions'' section, defining what the protocol should be able to do. Next
come the ``Acceptance criteria'', containing descriptions of requirements we
have on those interactions. Any important implementation details are detailed in
the ``Implementation Details'' section after that. Finally, the ``Tests''
section provides an exhaustive list of the way we plan to falsify whether we
should deploy. \\

Before getting into the protocol itself, let us explain how we will implement a
decentralized exchange, namely according to the automated market model. Rather
than having market makers and takers, such as is done in stock exchanges, the
users will be liquidity providers (LPs) and traders. The LPs provide liquidity
into a pool in exchange for a liquidity token, which is used as a `receipt' of
the amount of liquidity they provided. Traders can then swap against the pool.
The exchange rates for swaps are determined by a so-called invariant equation.
For example, the initial invariant equation used by Danaswap will be $x \cdot y
= k^2$. This means that the amount $\Delta y$ a trader gets in exchange for
adding $\Delta x$ is such that $(x + \Delta x) \cdot (y - \Delta y) = k^2 = x
\cdot y$. \\

When providing liquidity, $x \rightarrow x'$ and $y \rightarrow y'$ has the
effect that $k \rightarrow k'$. We now decide that $L \sim k$. This means that
the number of liquidity tokens the LP receives, $\Delta L$, follows the rule
$\Delta L / L = \Delta k / k$. \\

It seems useful to mention here why we decided to call the constant in the
invariant equation $k^2$, rather than writing for example $x \cdot y = C$. The
reason for this is that $k = \sqrt{x \cdot y}$ is proportional to the value in
the pool\footnote{
  There are two clear ways to see this:
  \begin{enumerate}
    \item Imagine doubling the balance of each asset in the pool. Clearly
      the value of the pool doubled, showing that the square root is the
      correct power law.
    \item Another way is unit analysis. Both $x$ and $y$ are expressed in
      the unit of some cryptocurrency, as is the value of the pool. Hence $x
      \cdot y$ is expressed as cryptocurrency unit squared, showing the need
      to take the square root.
  \end{enumerate}
}. Therefore, we can now make $L$ proportional to $k$ in transactions which add
or withdraw liquidity.

\section{Interactions}

\subsection{Initialize pool with liquidity}

This should create a pool with some initial balance for both assets, giving the
creator the appropriate number of liquidity tokens. Anyone can initialize a pool
through the UX. Each pool contains two assets and has its own liquidity token.
Once a pool is created between assets A and B, the UI no longer allows to create
a pool between these assets. \\

Note that the `admin' user is determined not by who creates pools, but by a set
of three pre-determined public keys, which are hardcoded into the scripts.

\subsection{Swap}

A user can exchange tokens at a rate defined by the constant product
invariant\footnote{
  Invariant equations are used in DEXes in order to represent supply and demand.
  With other words, invariant equations represent the idea that as the supply of
  $x$ increases, its value (price) decreases.
},
$x \cdot y = k^2$. This means that, ignoring fees, the user gets $\Delta y$ in
return for paying $\Delta x$, where $(x + \Delta x) \cdot (y - \Delta y) = k^2$.
\\

Note: No square roots ever need to be computed on-chain, since we only need to
verify that the invariant holds. This can be done in terms of $L^2$, not $L$.
This is the case across the entire protocol.

\subsection{Add liquidity}

Any user can add liquidity to a pool through the UX, minting that pool’s
liquidity tokens in return. The number of liquidity tokens minted is
proportional to the growth of the pool. Specifically, this means that $\Delta k
/ k = \Delta L / L$, where $k$ is the parameter in the equation $x \cdot y =
k^2$, $L$ is the number of liquidity tokens minted and $\Delta L$ is the number
of liquidity tokens minted again. This endpoint charges fees.

\subsection{Withdraw liquidity}

A user who owns liquidity tokens for a given pool, can burn them through the UX
in exchange for getting back a proportional part of the assets left in the pool.

\subsection{Withdraw liquidity in one asset class}

A user who owns liquidity tokens for a given pool, can burn them through the UX
in exchange for getting back liquidity in one token, following the invariant
equation $x \cdot y = k^2$. This endpoint charges the fees.

\subsection{Kill pool (admin)}

Any admin user can kill any pool. This disallows any transaction with the pool
except for the regular ``Withdraw liquidity'' endpoint. Even the ``Withdraw
liquidity in one asset class'' endpoint is no longer allowed.

\subsection{Collect admin fees (admin)}

The admin user can collect all the admin fees from a pool.

\subsection{List pools}

The UX show all Danaswap pools.

\subsection{Show pool info}

For each Danaswap pool, the UX shows the asset amounts in the pool, total amount
of liquidity tokens minted, exchange rates and price of the liquidity token.

\section{Acceptance criteria}

General rule for all transactions involving pools: No transaction has more than
one pool in the outputs. The goal of this rule is to simplify the space of
protocol attacks.

\subsection*{Initialize pool with liquidity}

\begin{itemize}
  \item When initializing a pool, no asset balances can be $0$
  \item Once a pool is initialized between assets $A$ and $B$, the UX disallows
    creating another pool between the same assets
    \todo{@Leo: This has no corresponding test descriptions yet.}
  \item Any user can initialize a pool
  \item Creating a pool leaves the user with up to $\sqrt{a_0 \cdot b_0}$ of the
    liquidity tokens of the newly formed pool
\end{itemize}

\subsection*{Swaps \& liquidity transactions} \label{dex-transactions}

First of all, as mentioned above, the protocol will charge fees on swaps, adding
liquidity and withdrawing liquidity in one asset class. In addition, we must
enforce the invariant equation, which is difficult due to rounding errors.
Therefore, we replace both problems by the following rule: After any
transaction, $x \cdot y / L^2$ has not decreased. This is because any
transaction should only increase the value of each liquidity token. In this
statement,
\begin{itemize}
  \item $x$, $y$ and $L$ come from the non-admin balance of the datum of the
    input UTXO
  \item $L' = L + \Delta L$, where $\Delta L$ is the amount of $L$ burned/minted
    in the transaction
  \item For balanced withdrawals, $x' = x - \Delta x$ where $\Delta x$ is the
    amount changed in the value of the pool. The non-admin balance changed the
    same amount, and the admin balance didn't change at all.
  \item For swaps, adding liquidity and withdrawals in one token, $x' = x +
    \Delta x$. Let $\delta x$ be the amount $x$ changed in the value of the
    pool. Then $\Delta x = 99.4\% \cdot \delta x$. In addition, $0.3\% \cdot
    \delta x$ is added to the admin balance (in the pool's datum).
\end{itemize}
This explains how the invariant equation is enforced\footnote{
  We would like to claim that $x \cdot y = k^2$ and $L \sim k$. However, this
  leads to rounding issues. Therefore, we instead require that $x \cdot y \leq
  k^2$ and $L \sim k$, meaning that the value per liquidity token can only
  increase in a given transaction. In short, this means that the value $x \cdot
  y / L^2$ is only allowed to increase through transactions.
} as well as well as solving the rounding problem and explaining how the admin
and non-admin fees will work. \\

For each of these transaction, acceptance criteria include
\begin{itemize}
  \item The value per liquidity token doesn't decrease
  \item Fees are handled correctly, as explained above
\end{itemize}

The other rule that all these transactions have in common, is that non-admin
balances $+$ admin balances is equal to the value in the pool.

\subsection*{Swap}

\begin{itemize}
  \item Small swaps create low slippage
  \item If the balances of the two assets in a pool are roughly equal, their
    exchange rate is roughly $1$
  \item Swaps selling $A$ to the pool in exchange for $B$ increase the exchange
    rate $A \rightarrow B$ and decrease the exchange rate $B \rightarrow A$
\end{itemize}

\subsection*{Add liquidity}

\begin{itemize}
  \item Any user can add liquidity
  \item When adding liquidity, the user receives all newly minted liquidity
    tokens
  \item When adding liquidity, the total number of liquidity tokens (in the
    pool's datum) is increased by the same amount as are tokens minted
\end{itemize}

\subsection*{Withdraw liquidity}

\begin{itemize}
  \item Any user who holds liquidity tokens for a pool, can withdraw liquidity
    out of that pool
  \item When withdrawing liquidity, the total number of liquidity tokens (in the
    pool's datum) decreases by the same amount as are tokens burned
\end{itemize}

\subsection*{Kill pool}

\begin{itemize}
  \item The admin users (i.e. he who can kill pools) is independent of who
    created the pool
  \item The admin users can kill pools
  \item Non-admin users cannot kill pools
  \item A pool that is killed, only supports the ``Withdraw liquidity'' endpoint
\end{itemize}

\subsection*{Collect admin fees}

\begin{itemize}
  \item Admin can collect the admin fees: The admin balances in the datum get
    set to zero, and paid out of the pool's value to the admin wallet
  \item Non-admin cannot collect admin fees
\end{itemize}

\subsection*{Pool info}

\begin{itemize}
  \item A list of pools can be queried
  \item When a new pool is initialized, the list of valid pools increases by one
  \item When an invalid pool is created, the list of valid pools increases by
    one
  \item When a pool is updated (kill, swap, add liquidity, withdraw liquidity,
    collect admin fees), the list of valid pools doesn't change
  \item For each valid pool, its pool info can be queried
  \item When a valid pool is updated, its pool info is updated appropriately:
    Killing a pool makes the pool info say it is killed, adding liquidity mints
    the same number of liquidity tokens as are added to the ``total number of
    liquidity tokens minted'' etc.
  \item Each pool is verifiable, i.e. any pool which has been initialized
    according to the pool validator script is considered a legitimate pool and
    vice versa
\end{itemize}

\section{Implementation details}

Firstly, each pool will be identified through an NFT, referred to as the Pool
ID. The goal is to make each pool identifiable, traceable and verifiable, in
that we can check if a UTXO at the pool script address was created through its
script rather than created elsewhere and send to the address. This pattern was
also used in the Hello Cardano Template (HCT), and will be copied from there. \\

Secondly, in order to run our DEX pools, we need a liquidity token for each
pool. This association between liquidity token and pool is achieved by encoding
the pool ID into the token name of the liquidity token. The currency-symbol is
the same for all liquidity tokens, to make liquidity tokens easily identifiable.
This means that all liquidity tokens will be minted through the same minting
policy script. \\

Lastly, our protocol consists of a number of scripts whose functionalities are
dependent on each other. Some of these dependencies are cyclic, making the
hashes of the scripts cyclicly dependent. We resolve this through the
\textit{Config UTxO} pattern.

\subsection{Config UTXO}

Naively, there are cyclic dependencies between the hashes of different scripts
involved in the protocol. This requires us to find the right hashes to fit
everything together, which is virtually impossible by construction of the
Cardano blockchain. An alternative solution is a dedicated on-chain component
which tracks the script hashes involved in the protocol in its datum. This
pattern is named a Config UTXO. It is identified by an NFT. Each script is then
able refer to the script hashes in the Config UTXO in order to verify the other
scripts, breaking any at-compiletime cyclic dependencies. \\

A drawback of the Config UTXO pattern is the greater transaction size and thus
higher transaction fees. We want to avoid this problem for all transactions
except for initializing a pool, because that doesn't happen very frequently.
Therefore our Config UTXO will contain only two hashes: the liquidity token
minting policy hash and the pool address validator script hash. The third hash,
namely the pool ID minting policy hash, will be passed as a parameter to the
other scripts. See Figure \ref{hashGraph} \\

\begin{figure}
  \scalebox{0.5}{\includegraphics{diagram}}
  \caption{
    In the diagram above blue arrows represent passing hashes by Config UTXO and
    red arrows represent passing hashes by script parameter.
  }
  \label{hashGraph}
\end{figure}

This leaves two remaining notes:
\begin{itemize}
  \item The Config UTXO should never change. This means that once created, it
    should be put at an address associated with a script which make any
    transaction fail.\footnote{
      Note that reference inputs don't require their validator scripts to be
      invoked.}
  \item The config UTXO needs an NFT to distinguish it from any invalid config
    UTXOs. This NFT will be generated through a boilerplate minting policy from
    the Hello Cardano Template.
\end{itemize}

\subsection{Pool ID Minting Policy}

Each pool will be identified by a NFT. This allows anyone to verify whether a
UTXO at the pool script address is valid, as well as tracing each pool. This
constitutes structurally the same minting policy as the one in
\href{https://github.com/ArdanaLabs/cardano-app-template/blob/master/onchain/src/HelloDiscovery.hs}{
  Hello Discovery}.
The only distinction with the Hello Discovery minting policy is that the Pool ID
policy will refer to the Danaswap Config UTXO.

\subsection{DEX Pool: Liquidity Token Minting Policy}

The liquidity token minting policy will allow any valid pool\footnote{
  A valid pool is any UTXO at the pool script address which holds an NFT minted
  by the Pool ID minting policy.
} to mint and burn its corresponding liquidity token. The liquidity token
minting policy needs to know which liquidity token belongs to which pool through the
Pool ID (NFT).

\subsection{DEX Pool}

The DEX pool script controls interactions that the pool allows. It is therefore
responsible for controlling swaps, liquidity deposits and withdrawals, pool
termination and admin fee collection. It is also responsible for ensuring that
pool ID tokens never leave its addresses because they could then be used to open
invalid pools. \\

When initializing a pool, $k_0^2 = x_0 \cdot y_0$ where $x_0$ and $y_0$ are the
initial balances of asset classes $X$ and $Y$. The admin user receives $L_0 =
k_0$ liquidity tokens. \\

Each pool datum consists of a number of parts:
\begin{itemize}
  \item The amount in each asset class ($X$ and $Y$) for the admin fees
    collected and the non-admin amounts.
    These amounts are subtracted from the pool's balances in any calculations
    involving the invariant equation.
  \item The number of liquidity tokens ever minted (minus burned)
  \item A boolean representing the pool's kill switch
\end{itemize}

The script hardcodes the following:
\begin{itemize}
  \item Admin fee
  \item Liquidity provider fee
  \item The three admin public keys
  \item The Config UTXO's ID NFT
\end{itemize}

To do this it enforces the following:
\begin{itemize}
  \item The admin and non-admin balances in the pool sum up to the balance in
    the pool's value, before and after the pool
  \item The output contains exactly one pool. This is done in order to ensure
    e.g. that two pools don't switch Pool IDs.
  \item The liquidity invariant $x \cdot y / L^2$ is non-decreasing before fees
  \item The input and output pool have a Pool ID NFT
  \item The ``liveness'' of the pool is not changed unless the redeemer is Kill
  \item The redeemer is valid \todo{@Brian: What does this mean?}
  \item When the redeemer action is Initialize: \todo{@Brian: Fill in this
    list.}
  \item When the redeemer action is Swap:
    \begin{itemize}
      \item The Admin fee is increased by $0.3\%$ of the output
      \item The pool increases by $0.3\%$ of the output
      \item No liquidity tokens are minted
      \item The input and output pools are both live
    \end{itemize}
  \item When the redeemer action is AddLiquidity, WithdrawLiquidity or
    WithdrawLiquidityInOneAssetClass
    \begin{itemize}
      \item Appropriate fees are paid
      \item The datum reflects the change in issued liquidity
      \item If the input pool is dead the redeemer action is WithdrawLiquidity
        (balanced)
    \end{itemize}
  \item When the redeemer action is Kill
  	\begin{itemize}
      \item The input pool is alive
  		\item The output pool is dead
  		\item Nothing else is changed in the pool's value or datum
  		\item At least one admin key signed the Tx
  	\end{itemize}
  \item When The redeemer action is Collect
  	\begin{itemize}
  		\item The admin fees in the output pool are zero
  		\item The non-admin fees balance stays the same
  		\item The output value remains correct for the datum
  		\item At least one admin key signed the Tx
  	\end{itemize}
\end{itemize}

\subsection{Pool ID minting policy}

The pool ID is a token used to track pools. In addition, it ensures that a given
pool has been opened legitimately. The pool ID's minting policy is responsible
for ensuring that it is only minted as part of opening a valid vault. To do so
it should enforce the following:

\begin{itemize}
  \item The config UTXO contains the right NFT as its ID
  \item Exactly one token is minted
  \item That token is sent to a pool
  \item The redeemer provides a seed UTXO
  \item The token name of the minted token is the hash of the seed UTXO
  \item The seed UTXO is spent in the transaction
  \item The output pool datum is valid
  \item The datum correctly reflects the number of liquidity tokens being minted
  \item At least one liquidity token is being minted
  \item The initial invariant is not zero ie. some liquidity is provided in each
    token
  \item The issued liquidity is exactly the square root of the initial value,
    $x_0 \cdot y_0$
\end{itemize}

NOTE: The UX will not allow creating pools between asset class pairs which
already have an associated pool. This is not enforced on-chain.

\subsection{Liquidity token minting policy}

Liquidity tokens are used to represent ownership of liquidity in a pool. Their
minting policy is responsible for ensuring that they are only minted during
liquidity interactions with a pool and during pool initializations. \\

Since the Pool's validator script and the pool ID token minting policy already
enforce the logic regarding how much should be minted, the liquidity token
minting policy doesn't have to do much. Its main responsibility is enforcing
that some pool is involved either by being opened or an existing pool being
used, and that the liquidity tokens being created correspond to the pool being
used. The way we keep track of which liquidity tokens correspond to which pools
is by having the token name be the ID of the pool the liquidity tokens belong
to. \\

To do that it enforces the following:
\begin{itemize}
	\item The liquidity tokens being minted all have the same token name/pool id
	\item The redeemer is either Initialize or Pool
  \item When the redeemer is Initialize, a pool ID token with the same ID is
    minted
  \item When the redeemer is Pool, there is an input with a pool ID token with
    the same ID
\end{itemize}

\subsection{Initializing the protocol}

The first step to initialize the protocol, is to mint an NFT, which will be
considered the `master' identifier of the protocol, identifying the Config UTXO.
Next, using the name of this NFT, we can calculate the hash associated with the
three other scripts (pool validator script, liquidity token minting policy and
Pool ID minting policy). Finally, a Config UTXO is initialized with the `master'
NFT and a datum that contains the two hashes, in that order: Pool script and
pool ID script.

\subsection{Queries}

Pools can be listed by getting all the UTXOs at the pool addresses and filtering
out pools which don't have valid ids. All the pool info for a given pool
can be gotten by looking at the datum and value of the UTXO.

\section{Tests}

\todo{List off-chain tests.}

Note: The off-chain tests also check that the on-chain code allows valid
transactions.

\todo{Should we write Apropos-Tx tests?}

\todo{List CLI tests.}

\todo{List browser tests.}

\subsection*{NFT}

The tests for the NFT minting policy which creates the Config UTXO ID.
\begin{itemize}
	\item When minting an NFT with the seed UTxO as an input, validation passes.
	\item When minting an NFT with the seed UTxO as a reference input, validation fails.
	\item When minting an NFT more than once, validation fails.
	\item When burning an NFT, validation fails.
\end{itemize}

\subsection*{Config UTxO}

\begin{itemize}
	\item When spending the Config UTxO, validation fails.
\end{itemize}

\subsection*{Liquidity Token Minting Policy tests}

\begin{itemize}
	\item When minting a different pool's liquidity token on pool open, validation fails.
	\item When minting multiple liquidity token types on pool open, validation fails.
	\item When minting liquidity tokens for a different pool than the pool being spent, validation fails.
	\item When burning liquidity tokens from a different pool than the pool being spent, validation fails.
	\item When minting liquidity tokens for a pool at an address that is not the Pool Address Validator, validation fails.
\end{itemize}

\subsection*{Pool id Token Minting Policy tests}

For this section a "valid pool open" refers to a Tx where

\begin{itemize}
	\item A pool is opened with 100 each of two dummy tokens.
	\item The admin fees are set to 0
	\item the issued liquidity is $\sqrt{A_0 B_0}$ (in this case 100)
	\item the pool liveness is set to true
	\item The transaction should reference the valid config utxo
	\item A seed utxo should be chosen
	\item The seed utxo is spent
	\item The seed utxo is passed to the pool id minting policy via the redeemer
	\item the pool id is the hash of the seed utxo
	\item The number of issued liquidity tokens set in the pool's UTxO datum matches the number of liquidity tokens minted from the liquidity token minting policy.
	\item The pool id NFT is sent to the pool
\end{itemize}

Any Tx mentioned in these tests should be the same as the valid pool open
except for any contradicting details stated in that test.

\begin{itemize}
	\item Non admin can initialize a pool.
		\\ To test this we should validate a valid pool open
	\item Can't under pay for liquidity on open
		\\ To test this we should fail to validate a 2 txs
			\begin{itemize}
				\item One where:
					We only provide 90 of each token.
					but report minting 100
					and mint 100
				\item One where:
					we only provide 90 of each token
					and in the datum report minting 90
					but actually mint 100
			\end{itemize}
	\item Can't open pool with 0 liquidity
		\\ To test this we should fail to validate a tx where
			both amounts and the liquidity are zero
	\item Can't open pool with 0 of either exchange token
		\\ To test this we should fail to validate two Txs
		one where each currency amount is zero
	\item A utxo without the config NFT can't be used to open a pool
		\\ To test this we should
		produce an identical config utxo without the NFT
		and then fail to validate a tx
		where the invalid config utxo is referenced instead
	\item Can't mint more than 1 token
		\\ To test this we should fail to validate a tx where
		two pool id tokens are minted
	\item Can't send identifier token to non-pool address
		\\ To test this we should fail to validate a tx where
		the pool id NFT is kept by the user
	\item Can't mint pool with wrong pool id
		\\ To test this we should fail to validate a tx where
		the token name of the minted id is ``aaaa"
	\item Can't mint without spending seed Tx
		\\ To test this we should fail to validate a tx where
		the seed utxo is not spent
\end{itemize}

\subsection*{Pool Address Validator tests}

\begin{itemize}
	\item Can't remove pool id from address during any type of transaction
	\item Can't kill a pool with a swap
	\item Can't kill a pool with a liquidity change
	\item can perform valid swaps
	\item can't underpay for swaps
	\item can perform valid liquidity changes
	\item can't perform underpaid liquidity changes
	\item admin can kill a pool
	\item non-admin can't kill a pool
	\item admin can't over extract fees
	\item liquidity can be extracted from a dead pool
	\item Can't steal pool ID tokens (this should consist of tests at each endpoint)
	\item Pool ids can't be swapped between two pools
  \item It shouldn't be possible to add liquidity and then withdraw it in order to essentially swap with
	\item Can't swap on a dead pool
	\item Can't add liquidity to a dead pool
	\item Can't make unbalanced withdrawals from a dead pool
    reduced fees
	\item Given a pool containing total issued liquidity tokens of quantity $t$ of $T$ and asset
	      pair of quantities $a$ of $A$ and $b$ of $B$, when the user provides quantities $\Delta a$ and $\Delta b$
  	      and mints a quantity of liquidity tokens less than or equal to
	      $T \cdot \frac{\sqrt{ (a + \Delta a) \cdot (b + \Delta b)} - \sqrt{a \cdot b}}{\sqrt{a \cdot b}}$
	      , validation passes.
	\item Given a pool containing total issued liquidity tokens of quantity $t$ of $T$ and asset
	      pair of quantities $a$ of $A$ and $b$ of $B$, when the user provides quantities $\Delta a$ and $\Delta b$
	      and mints a quantity of liquidity tokens greater than
              $T \cdot \frac{\sqrt{ (a + \Delta a) \cdot (b + \Delta b)} - \sqrt{a \cdot b}}{\sqrt{a \cdot b}}$
	      , validation fails.
	\item Given a pool containing total issued liquidity tokens of quantity $t$ of $T$ and asset pair of quantities
	      $a$ of $A$ and $b$ of $B$, when the user burns
	      their liquidity token quantity $\Delta t$ and receives a quantity of $A$ less than or equal to
	      $ a \cdot \Delta t/t $ and quantity of $B$ less than or equal to $ a \cdot \delta t/t $
	      , validation passes.
	\item Given a pool containing total issued liquidity tokens of quantity $t$ of $T$ and asset
	      pair of quantities $a$ of $A$ and $b$ of $B$, when the user burns
	      their liquidity token quantity $\Delta t$ and receives a quantity of $A$ greater than
	      $ a \cdot \Delta t/t $ or (||) quantity of token $B$ greater than or equal to $ a \cdot \delta t/t $
	      , validation fails.
\end{itemize}

\subsection*{Query tests}

The following intended behaviors need to work:
\begin{itemize}
	\item New pools show up in queries
	\item New pools without pool id tokens do not show up in queries
	\todo{Do we need to worry about a large number of invalid pools slowing down
	queries? If so should we add an interaction to remove them?}
	\item Queries accurately reflect how the pool was setup
	\item Queries accurately reflect the change from a swap
	\item Queries accurately reflect liquidity changes
	\item Queries accurately reflect pool kills
	\item A swap in the $A \rightarrow B$ direction increases the price of $B$
	\item Exchange rate query is accurate for small exchanges
\end{itemize}

Additional tests:
\begin{itemize}
	\item When adding liquidity, the number in the pool's datum representing the
    total amount of liquidity minted goes up by the same amount as the actual amount minted
\end{itemize}

Note: We should improve on the template by making better tools for checking that tests fail for the right reason.

\end{document}
