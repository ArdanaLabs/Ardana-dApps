\documentclass{article} % Use the custom invoice class (invoice.cls)
\usepackage{todonotes}
\setuptodonotes{inline}
\setlength{\parindent}{0em}

\title{dUSD Test Plan}
\begin{document}

\maketitle

\section{Introduction}

\textbf{The purpose of this document is to let the dUSD team decide whether we are ready to deploy.}
It is a living document that serves as a forcing function for discussions
around, and agreement on, decisions. \\

The document starts with the ``Interactions'' section, a user-centric
perspective on what service we would like to offer.
Next come descriptions of requirements we have on those interactions in the
``Acceptance criteria'' section.
Any important implementation details that will be relevant for building such a
system are detailed in the section after that.
Finally, the ``Tests'' section provide a non-exhaustive list of the way we plan
to falsify whether we should deploy. \\

Note: Every subsection that contain `TBD' in the title is not fully agreed upon
yet.
The other sections are considered law.

\section{Interactions}

\subsection{Discover market prices (MVP version)}

For the MVP version, we implement a simplified, centralized market price
discovery mechanism.
This mechanism involves only a single on-chain component, called the Price
Module (PM).
This module contains $48$ hours of price data, aggregated off-chain and put
on-chain by admin.

\subsubsection{Price module: Write}

Admin can update the price information in the PM's datum by adding one more
price point.
When doing so, any price points older than $48$ hours are removed.

\subsubsection{Price module: Read}

Anyone can use the price module as a read-only input UTXO to their transactions
in order to discover the last $48$ hours of price information, one price point
per hour.

% The V2 version is left here in comments, for now.
% \subsection{Discover market prices (V2)}
% 
% A price oracle is a timeline of UTXOs\footnote{
%   In the blockchain industry, people often talk about concepts like price
%   oracles and DEX pools. However, within the EUTXO system built by Cardano, the
%   fundamental concept we are dealing is the UTXO. This means that we are dealing
%   with a `state' of a price oracle as the fundamental concept, not the price
%   oracle itself. Now, Plutus scripts like price oracles, between being created
%   and destroyed, basically only allow to be updated and read. This means that
%   one can draw a `timeline' through a series of UTXO, each one disappearing into
%   a transaction that produces the next. This set of UTXOs is what is referred to
%   as ``the price oracle''.
% }
% which stores the price information of a given asset (e.g. ADA) in terms of a
% reference currency (USD).
% Each price oracle will store $48$ hours worth of price data, and be updatable
% once every few minutes.
% A number of price oracles will be created for each asset of interest.  The
% person creating each price oracle will connect it to a price feed coming from
% some CEX/DEX.
% Next, an Oracle Security Module (OSM) will be implemented.
% This is another ``timeline of UTXOs'', which allows admin to select a number of
% price oracles as trusted price feeds and combines their prices in a secure way.
% The resulting price is exposed to the blockchain ecosystem.
% 
% Note: In this document, off-chain entities containing information about the
% price will be referred to as price feeds (not price oracles), e.g. the Coinbase
% website.
% 
% Note: We will build price oracles both for ADA and dUSD.
% 
% \subsubsection{Price oracle: Read}
% 
% The OSM can read all the pricing data stored in the price oracle.
% 
% \subsubsection{Write}
% 
% The owner/creator of the price oracle can write a new price value to the oracle.
% This endpoint can be called once per hour.
% 
% \subsection{Oracle security module}
% 
% \subsubsection{Read the `current' price}
% 
% Anyone can use the OSM as a read-only input UTXO to discover the price
% information of the last $48$ hours of some currency (e.g. ADA or dUSD relative
% to USD).
% 
% \subsubsection{Update}
% 
% The OSM is updated once per hour.
% Anyone can do this by connecting the OSM to all accepted price oracles.
% 
% \subsubsection{Update list of accepted price oracles}
% 
% Admin/governance can update the list of price oracles accepted by the OSM.
% 
\subsection{Vault (TBD)}

\subsubsection{Collect vault information}

As a user, I can collect information on my vaults, i.e. the vaults I created.

\subsubsection{Initialize vault}

As a user, I can open a new vault.

\subsubsection{Deposit}

As a user, I can deposit ADA into a vault, so that I may take out a loan of dUSD
(see below).

\subsubsection{Withdrawal}

As a user, I can withdraw ADA from a vault.

\subsubsection{Take out loan}

As a user, I can take out a loan in dUSD against the collateral I put into one
of my vaults.

\subsubsection{Pay back loan}

As a user, I can pay back (part of) a loan in dUSD I created in one of my
vaults. In doing so, I am paying a stability fee to the buffer.

\subsubsection{Liquidate vault}

Any user can detect that a vault is in an unhealthy state, and buy part of the
collateral in the vault at a discount (compared to market prices).

If after liquidating all the collateral, a vault has remaining debt, the vault
is destroyed and the debt with it.
This is an exceptional situation to be avoided, since it endangers the idea
behind the protocol.

\subsection{Buffer (TBD)}

All profits will be kept in a UTXO referred to as the buffer.
In addition to holding the profits, the buffer will be responsible for setting
the protocol-wide parameters, and for enacting debt and surplus auctions.

\subsubsection{Trigger surplus auction (buy back DANA)}

\subsubsection{Trigger debt auction (sell off DANA)}

\subsubsection{Set liquidation fee}

\subsubsection{Set liquidation ratio}

\subsubsection{Set minimum collateralization ratio}

\todo{Do we want to have an on-chain enforced minimum collateralization ratio?}

\subsubsection{Set stability fee}

\section{Acceptance criteria}

\subsection{Price module (MVP)}

Off-chain and on-chain criteria:
\begin{itemize}
  \item Only admin can create a price module
  \item Only admin can write to the oracle
  \item Writing a price value removes price values older than 48 hours from the
    datum
%   \item Writing the price value cannot be done more than once per hour
  \item Reading the price data after writing a new price point successfully,
    works
  \item Anyone can read the price data
\end{itemize}

Price feed bot:
\begin{itemize}
  \item Always collects information from at least three sources successfully
  \item Running the systemd service makes the UTXO get updated by at least three
    sources at least once every two hours
\end{itemize}

% The version 2 price discovery mechanism is stored in these comments:
% \subsection{Price oracle}
% 
% \begin{itemize}
%   \item Only the owner can write to the oracle
%   \item Anyone can read the price data
%   \item Writing a price values removes outdated price values (older than 48
%     hours) from the datum.
% \end{itemize}
% 
% \subsection{Oracle security module}
% 
% \begin{itemize}
%   \item Admin (and only admin) can update the list of accepted price oracles
%   \item The OSM's price information cannot be updated without linking to all
%     accepted price oracles
%   \item The price oracles are read-only inputs to the ``update OSM'' endpoint
%   \item Reading the current price after updating the OSM's price information and
%     waiting for the given time delay, leads to the predicted `current' price.
%     This is the median over all price oracles of the price averaged over the
%     values in the last hour. (This smearing prevents flash drops on some
%     exchanges from making the OSM's result crash and many vaults liquidate.)
% \end{itemize}
% 
\subsection{Vaults (TBD)}

Each user should have access to a website at "www.ardana-vaults.com". This
website must be able to connect to their wallet to find out who they are and
allow them to set up vaults and interact with their existing vaults. \\

Decisions to be made:
\begin{itemize}
  \item Do we want the interface to be a website or an app?
  \item Which domain name will the website live at?
\end{itemize}

\subsubsection{Collect information on the user's vaults}

The vault website looks up the requested information and displays it in an
aesthetic and clear overview. \\

Requested information: How many vaults does the user have, how much collateral
and debt do they each have, what are their collateralization ratios, how
dangerous is it to have that collateralization ratio? \\

Decisions to be made:
\begin{itemize}
  \item Is the list of requested information correct? Anything to be added,
    removed or updated?
\end{itemize}

\subsubsection{Initialize}

The vault website should have a button to open a new vault.

\subsubsection{Deposit}

The vault website has a form that allows each user to pick one of their vaults
and deposit ADA into it. \\

Restrictions:
\begin{itemize}
  \item You have at least the amount of ADA in your wallet that you would like
    to deposit.
\end{itemize}

Decisions to be made:
\begin{itemize}
  \item Do we want to implement a maximum amount of collateral? MakerDAO has
    this, not sure why.
  \item Do we want to implement a mininum amount of collateral? If so, is this
    in total per vault, or per deposit?
\end{itemize}

Questions:
\begin{itemize}
  \item How will we figure out from which UTXO in your wallet to take the ADA?
\end{itemize}

\subsubsection{Withdrawal}

The vault website has a form that allows each user to pick one of their vaults
and withdraw ADA from it. \\

Restrictions:
\begin{itemize}
  \item Withdrawing this amount of ADA from that vault doesn't sink its
    collateralization ratio below the minimum collateralization ratio.
\end{itemize}

\subsubsection{Take out loan}

The vault website has a form that allows a user to pick one of their vaults and
take out a dUSD loan in it, against his collateral. \\

Restrictions:
\begin{itemize}
  \item This transaction doesn't make the collateralization ratio drop below the
    minimum collateralization ratio.
\end{itemize}

\todo{What should the minimum collateralization ratio be?}
\todo{Should the minimum collateralization ratio be different from the
liquidation ratio??}
\todo{What should the minimum amount be one can take out as a loan? Is this per
vault or per loan transaction?}

\subsubsection{Pay back loan}

The vault website has a form that allows a user to pick one of their vaults and
pay back a (part of a) dUSD loan in it. \\

Restrictions:
\begin{itemize}
  \item The amount of dUSD paid back by the user is equal to or smaller than the
    amount of dUSD debt in the vault.
  \item The stability fee for the part of loan that is paid back, is sent to the
    buffer as community profit
\end{itemize}

Design decisions:
\begin{itemize}
  \item What should the minimum collateralization ratio be?
  \item Should the minimum collateralization ratio be different from the
    liquidation ratio??
  \item Should there be a minimum amount one can take out as a loan? If so, is
    this per vault or per loan transaction?
\end{itemize}

\subsubsection{Liquidate someone else's vault}

...

\subsection{Buffer (TBD)}

One important acceptance criterium is that the protocol can be updated without
requiring users to update their vaults or dUSD becoming a new currency.

Any updates to the protocol will be decided on by admin, and later governance.

\subsubsection{Trigger surplus auction (sick vault)}

\subsubsection{Trigger surplus auction (buy back dana)}

\subsubsection{Trigger debt auction}

\subsubsection{Set liquidation fee}

\subsubsection{Set liquidation ratio}

\subsubsection{Set minimum collateralization ratio}

\subsubsection{Set stability fee}

\subsubsection{Set list of price oracles to consider in OSM}

\subsubsection{Set the time interval the OSM waits before using a price update}


\section{Implementation details}

This section discusses what we need in order to implement the interactions
mentioned above.

\subsection{Price module (MVP, TBD)}

The price module is a UTXO which can only be created by admin.
It contains in its datum a map from timestamps (POSIX time) to the price value.
An off-chain bot will be written which aggregates price information from at
least five sources (CEXes and DEXes) and applies the appropriate calculations on
this information in order to conclude what the most accurate current price is.
This current price will then be submitted as a transaction to the price module
using one of the admin keys.

Off-chain bot:
\begin{itemize}
  \item Before collecting any price information, the script reads out the
    relevant UTXO's datum and checks if it is time yet to submit an update. If
    not, it will terminate (successfully) right away.
  \item Haskell script that collects information and submits a transaction
  \item The script is ran through a systemd service, running it once every five
    minutes
\end{itemize}

Price module:
\begin{itemize}
  \item The on-chain code will be written in Plutarch
  \item The off-chain code will be built as a PAB
\end{itemize}

Note: The off-chain bot will combine the price feeds through two operations:
\begin{enumerate}
  \item For each price feed, average the price over the values from the last
    hour
  \item Take the median of the resulting values
\end{enumerate}

% \subsection{Price oracle}
% 
% Each price oracle will carry an NFT used as a unique identifier.
% In addition, each price oracle contains the public key associated with the
% wallet which created it, so that only the creator of the oracle can update it.
% The price oracle can be updated once every hour.
% We will start with $3-5$ price oracles. \\
% 
% The OSM can be updated once every hour, and this transaction can be created by
% anyone.
% When it does, it pulls in the price information from all accepted price oracles
% and applies two calculations to that feed.
% First, it only retains the $70\%$ of price oracles that have most recently been
% updated, to assure obtaining the most recent price available.
% Secondly, it takes the median of the resulting price values.
% 
% \todo{How will price oracle updates and the OSM updates be rewarded? With a
% slight fee every time, coming from the profits stored in the buffer? Or should
% using a price oracle or OSM cost money, also to dUSD protocol users?}
% 
% If the OSM has not been updated for at least two hours, it stops spreading price
% information, thereby shutting down vault liquidations as well as creating new
% loans until the OSM is updated.
% This is a measure which will only kick in in the case of very serious resource
% contention, in which case once the system starts up again, we want to ensure
% that updating the OSM is the first transaction that happens.
% 
% \todo{Should the price oracles and the OSM all be combined into one UTXO, or
% be split into separate UTXOs?}
% 
\subsection{Vault}

Liquidation only becomes allowed when both the current price and the price of an
hour ago state that the current collateralization ratio is below the liquidation
ratio.
This is a measure taken to allow vault owners a chance to add collateral or pay
back some of the debt whenever the price of the collateral dropped.

\section{Tests}


% \section{Remarks}
% 
% - Test for Cardano network congestion
% - Test what happens if someone hacks one or more price oracles
% - Other attack vectors which are not specific to our application, should be
%   listed and tested for


\end{document}
