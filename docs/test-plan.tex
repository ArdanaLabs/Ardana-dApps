\documentclass{article} % Use the custom invoice class (invoice.cls)
\usepackage{hyperref}
\usepackage{todonotes}
\setuptodonotes{inline}
\setlength{\parindent}{0em}

\title{dUSD Test Plan}
\begin{document}

\maketitle

\section{Introduction}

\textbf{The purpose of this document is to let the dUSD team decide whether we are ready to deploy.}
It is a living document that serves as a forcing function for discussions
around, and agreement on, decisions. \\

The document starts with the ``Interactions'' section, a user-centric
perspective on what service we would like to offer.
Next come descriptions of requirements we have on those interactions in the
``Acceptance criteria'' section.
Any important implementation details that will be relevant for building such a
system are detailed in the section after that.
Finally, the ``Tests'' section provide a non-exhaustive list of the way we plan
to falsify whether we should deploy. \\

Note: Every subsection that contain `TBD' (``to be determined'') in the title is
not fully agreed upon yet.
Every subsection containing `TBC' (``to be confirmed'', by Ryan) in the title is
mostly agreed upon, but still requires last confirmation.
The other sections are considered law.

\section{Interactions}

\subsection{Discover market prices}

For the MVP version, we implement a simplified, centralized market price
discovery mechanism.
This mechanism involves only a single on-chain component, called the Price
Module (PM).
This module contains $48$ hours of price data, aggregated off-chain and put
on-chain by admin.

\subsubsection{Price module: Write}

Admin can update the price information in the PM's datum by adding one more
price point.
When doing so, any price points older than $48$ hours are removed.

\subsubsection{Price module: Read}

Anyone can use the price module as a read-only input UTXO to their transactions
in order to discover the last $48$ hours of price information, one price point
per hour.

% The V2 version is left here in comments, for now.
% \subsection{Discover market prices (V2)}
% 
% A price oracle is a timeline of UTXOs\footnote{
%   In the blockchain industry, people often talk about concepts like price
%   oracles and DEX pools. However, within the EUTXO system built by Cardano, the
%   fundamental concept we are dealing is the UTXO. This means that we are dealing
%   with a `state' of a price oracle as the fundamental concept, not the price
%   oracle itself. Now, Plutus scripts like price oracles, between being created
%   and destroyed, basically only allow to be updated and read. This means that
%   one can draw a `timeline' through a series of UTXO, each one disappearing into
%   a transaction that produces the next. This set of UTXOs is what is referred to
%   as ``the price oracle''.
% }
% which stores the price information of a given asset (e.g. ADA) in terms of a
% reference currency (USD).
% Each price oracle will store $48$ hours worth of price data, and be updatable
% once every few minutes.
% A number of price oracles will be created for each asset of interest.  The
% person creating each price oracle will connect it to a price feed coming from
% some CEX/DEX.
% Next, an Oracle Security Module (OSM) will be implemented.
% This is another ``timeline of UTXOs'', which allows admin to select a number of
% price oracles as trusted price feeds and combines their prices in a secure way.
% The resulting price is exposed to the blockchain ecosystem.
% 
% Note: In this document, off-chain entities containing information about the
% price will be referred to as price feeds (not price oracles), e.g. the Coinbase
% website.
% 
% Note: We will build price oracles both for ADA and dUSD.
% 
% \subsubsection{Price oracle: Read}
% 
% The OSM can read all the pricing data stored in the price oracle.
% 
% \subsubsection{Write}
% 
% The owner/creator of the price oracle can write a new price value to the oracle.
% This endpoint can be called once per hour.
% 
% \subsection{Oracle security module}
% 
% \subsubsection{Read the `current' price}
% 
% Anyone can use the OSM as a read-only input UTXO to discover the price
% information of the last $48$ hours of some currency (e.g. ADA or dUSD relative
% to USD).
% 
% \subsubsection{Update}
% 
% The OSM is updated once per hour.
% Anyone can do this by connecting the OSM to all accepted price oracles.
% 
% \subsubsection{Update list of accepted price oracles}
% 
% Admin/governance can update the list of price oracles accepted by the OSM.
% 
\subsection{Vault}

\subsubsection{Collect vault information}

As a user, I can collect information on my vaults, i.e. the vaults I created.

\subsubsection{Initialize vault}

As a user, I can open a new vault.

\todo{Do we want to make it possible to combine multiple transactions into one,
to help users avoid gas fees? \\
Proposal: No, this is too complex for an MVP, and once Orbis launches, gas fees
will be very low.}

\subsubsection{Deposit}

As a user, I can deposit ADA into a vault, so that I may take out a loan of dUSD
(see below).

\subsubsection{Withdrawal}

As a user, I can withdraw ADA from a vault, as long as my collateralization
ratio is above the liquidation ratio.
In the frontend, a "minimum collateralization ratio" will be built in that
prevents users from transacting with their vault in such a way that their
collateralization ratio drops below the minimum collateralization ratio.

\todo{Do we want a minimum collateralization ratio to protect our users? Where
will we implement it? \\
Proposal: Yes, since there are regular reports of people starting to use the
MakerDAO system as experimentation, and losing thousands of dollars on day one.
We will implement this in the frontend, not on-chain or off-chain.
This allows anyone to circumvent the child safety lock rather easily by using
our off-chain PAB without going through our frontend.}

\subsubsection{Take out loan}

As a user, I can take out a loan in dUSD if I have enough collateral.
There is a minimum size for dUSD loans, called the ``debt floor''.

\subsubsection{Pay back loan}

As a user, I can pay back (part of) a loan in dUSD I created in one of my
vaults.
In doing so, I am paying a stability fee to the buffer.

\subsubsection{Liquidate vault}

A vault is considered in an unhealthy state if its collateralization ratio is
below the liquidation ratio, both according to the current price and the price
from one hour ago.
The reason for this last part (using two prices) is that the vault owner should
get an hour after a new price gets published to fix his vault before allowing
liquidators to jump into action.
In addition, if a vault becomes sick through a very temporary ADA crash, but
recovers within an hour, there is no reason to liquidate. \\

Any user can detect that a vault is in an unhealthy state, and buy part of the
collateral in the vault at an admin-set discount rate compared to the market
prices of both ADA and dUSD.
This discount rate is called the liquidation discount, and will initially be
$3\%$.
Let us go through an example to explain how liquidation works.
When buying e.g. $100$ USD worth of ADA, the ADA is offered at a price of $97$
USD worth of dUSD.
Anyone can come in and buy a part of the ADA at this discounted exchange
rate\footnote{
  There are two reasons we allow liquidators to buy a part of the collateral,
  rather than forcing them to buy all of it.
  \begin{enumerate}
    \item Not forcing liquidators to buy all collateral makes the system less
      complex. Once a (possibly partial) liquidation has occured, the vault is
      still just a vault. If the collateralization ratio is still too low, a
      second person can come along and liquidate some more. If not, there's no
      reason a second person should be able to liquidate further.
    \item Forcing liquidators to buy all collateral would slow down liquidation
      arbitrage, since it requires one liquidity arbitrageur to have enough dUSD
      to buy all the collateral at once.
      Liquidation arbitrage is something we want to stimulate.
      It also unnecessarily assures punishing the vault owner very harshly.
  \end{enumerate}
}.

As time goes on, there are two options:
\begin{enumerate}
  \item By selling collateral at a discount, the collateralization ratio rises
    above the liquidation ratio, and the vault returns to a healthy state.
  \item All the collateral is sold but some dUSD debt remains in the vault.
    This is an exceptional situation to be avoided, since it endangers the idea
    behind the protocol.
    It that can only be caused by a significant ADA price drop where liquidators
    don't act on time.
    The solution is dissolving the vault, and thereby the dUSD debt.
\end{enumerate}

In addition to the liquidation discount given to liquidators, the vault owner
will be charged a liquidation fee for allowing his vault to become unhealthy.
The liquidation fee is initially set to $10\%$.
This means that only a part of the dUSD paid by the liquidators (in the initial
setup, $90\%$), is used to pay back debt, burning the rest.

\subsection{Buffer}

The buffer is a UTXO responsible for setting the protocol-wide parameters,
holding the protocol's profits and launching surplus and debt auctions.

\subsubsection{Trigger surplus auction (buy back DANA)}

A surplus auction is an auction where the buffer mints a certain amount of dUSD,
and sells it off in exchange for DANA tokens.
The DANA tokens are stored in the buffer.
The goal is to put more dUSD on the open market, increasing the supply and
therefore decreasing its price, while taking DANA tokens off the market,
increasing their price and rewarding governance holders. \\

Anyone can trigger a surplus auction when the dUSD price module claims that the
dUSD price lies above $1.01$ USD.

\subsubsection{Trigger debt auction}

A debt auction is the opposite of a surplus auction:
It is an auction where the buffer sells some of the DANA tokens it stores in
exchange for dUSD, and burns the dUSD.
The goal is to take some dUSD off the open market, decreasing the supply and
therefore increasing its price.
The result is that DANA tokens are put back on the market, decreasing their
price. \\

Anyone can trigger a debt auction when the dUSD price module claims that the
dUSD price lies below $0.99$ USD.

\subsubsection{Set liquidation discount}

The admin user can set the liquidation discount, through running a `curl'
command on the Ardana Tenant.

\subsubsection{Set liquidation fee}

The admin user can set the liquidation fee, through running a `curl' command on
the Ardana Tenant.

\subsubsection{Set liquidation ratio}

The admin user can set the liquidation ratio, through running a `curl' command
on the Ardana Tenant.

\todo{Do we want the liquidation ratio to be changeable within the protocol, or
only when the protocol is updated? \\
Proposal: We do not want it to be changeable.
It would give admin/governance a dangerous amount of power, and bad incentives.
In addition, the liquidation ratio should only change as the collateral types
(ADA) become more or less stable, which is a process that takes many months,
whereas the dUSD protocol will be updated regularly in the beginning.
This decision should be reviewed at each protocol update.}

\subsubsection{Set stability fee}

The admin user can set the stability fee (in percentage per year), through
running a `curl' command on the Ardana Tenant.

\subsubsection{Set debt floor}

The admin user can set the debt floor (in dUSD), through running a `curl'
command on the Ardana Tenant.

\subsection{Update mechanism for the protocol (TBD)}

One important acceptance criterium is that the protocol can be updated without
requiring users to update their vaults or dUSD becoming a new currency.

Any updates to the protocol will be decided on by admin, and later governance.

\section{Acceptance criteria}

\subsection{Price module}

Off-chain and on-chain criteria:
\begin{itemize}
  \item Only admin can create a price module
  \item Only admin can write to the oracle
  \item Writing a price value removes price values older than 48 hours from the
    datum
%   \item Writing the price value cannot be done more than once per hour
  \item Reading the price data after writing a new price point successfully,
    works
  \item Anyone can read the price data
\end{itemize}

Price feed bot:
\begin{itemize}
  \item Always collects information from at least three sources successfully
  \item Running the systemd service makes the UTXO get updated by at least three
    sources at least once every two hours
\end{itemize}

% The version 2 price discovery mechanism is stored in these comments:
% \subsection{Price oracle}
% 
% \begin{itemize}
%   \item Only the owner can write to the oracle
%   \item Anyone can read the price data
%   \item Writing a price values removes outdated price values (older than 48
%     hours) from the datum.
% \end{itemize}
% 
% \subsection{Oracle security module}
% 
% \begin{itemize}
%   \item Admin (and only admin) can update the list of accepted price oracles
%   \item The OSM's price information cannot be updated without linking to all
%     accepted price oracles
%   \item The price oracles are read-only inputs to the ``update OSM'' endpoint
%   \item Reading the current price after updating the OSM's price information and
%     waiting for the given time delay, leads to the predicted `current' price.
%     This is the median over all price oracles of the price averaged over the
%     values in the last hour. (This smearing prevents flash drops on some
%     exchanges from making the OSM's result crash and many vaults liquidate.)
% \end{itemize}
% 
\subsection{Vault (TBC)}

Each user should have access to a website at "www.ardana-vaults.com". This
website must be able to connect to their wallet to find out who they are and
allow them to set up vaults and interact with their existing vaults.

Acceptance criteria:
\begin{itemize}
  \item Anyone can create a vault, through a button on the website
  \item Anyone can create multiple vaults
  \item Vaults owned by the same person are independent, i.e. one of the
    person's vaults being sick, doesn't influence his other vaults
  \item There is a page that automatically connects to the user's wallet and
    displays information about his vaults: How many vaults, and for each vault
    the amount of collateral, debt, collateralization ratio, and how healthy the
    vault is.
  \item The vault overview is correct, clear and aesthetic
  \item Users can deposit, withdraw, take out a loan and pay back a loan
    (through forms on the website) only if the collateralization ratio doesn't
    drop below the liquidation ratio\footnote{
      In the UX (frontend only!), the collateralization ratio is banned from
      dropping under the minimum collateralization ratio.}
  \item No transaction can update the timestamp in the vault in a way that's
    inaccurate by more than $12$ hours
  \item Users cannot take out a loan or pay one back that leaves the dUSD debt
    above $0$ but below the debt floor\footnote{
    The puprose of the debt floor is to ensure good debt is issued against a
    vault to incentivize a liquidator to liquidate it should it become
    under-collateralized.}
% Comment for Version 2:
%   \item Users cannot take out a loan that leaves the dUSD debt above the debt
%     ceiling\footnote{
%     The debt ceiling has two purposes:
%     \begin{enumerate}
%       \item Deprecating collateral types by lowering the debt ceiling to below
%         the current amount of issued debt
%       \item Stopping OSM timing attacks, where vault owners exploit the OSM time
%         delay during a price crash, issuing a massive amount of debt thus
%         cashing out on worthless collateral
%     \end{enumerate}}
% 
% Debt ceilings are not useful against OSM timing attacks, since users can
% simply create multiple vaults.

  \item Users cannot withdraw or take out a loan that drops their
    collateralization ratio below the liquidation ratio
  \item When going through a sequence of transactions, stability fees are
    calculated correctly
  \item Only the owner of the vault can create transactions other than
    liquidation
  \item Anyone can liquidate a sick vault
  \item Liquidation is possibly if and only if the vault is in an unhealthy
    state
  \item One can buy as much collateral as is needed to make the vault healthy
    again, and not more
  \item During liquidation, collateral is bought at discount compared to the
    most current price from the price module, where the discount is the
    liquidation discount (set by admin)
  \item $10\%$ of the dUSD paid by the liquidator, is burnt rather than used to
    pay back the debt. This liquidation fee is meant to disincentivize users
    from allowing their vaults to become sick.
  \item The amount of dUSD paid back (when paying back a loan) is smaller than
    (or equal to) the loan
\end{itemize}

\todo{Which domain name will the website live at?}

\todo{Is the list of requested information correct? Anything to be added,
removed or updated?}

\todo{How much is the minimum loan (per vault, not per loan)? \\
Proposal: 100 dUSD.}

\todo{Frontend: How will we figure out from which UTXO in your wallet to take
the ADA?}

\todo{What should the liquidation ratio be?}

\todo{What should the minimum collateralization ratio be?}

\subsection{Buffer (TBC)}

\begin{itemize}
  \item Anyone can trigger a surplus or debt auction when the dUSD price
    deviates from USD by more than $1\%$
  \item When auctions can be triggered, they are triggered within minutes.
    Auctions cannot be triggered more than once per hour, to prevent
    overcorrecting the dUSD price.
  \item The starting amount of DANA in the buffer is enough to sustain the
    protocol through hardships with high probability
    \todo{What does this mean in practice? \\
    Proposal: Put in a decent amount at initialization, and set up an endpoint
    in the validator to add more later on.}
  \item The auctions are sized such that the dUSD price is controlled, i.e. it
    is eventually consistent with USD.
    This means that given any real world change in supply and demand, the dUSD
    price eventually converges back to within $1\%$ of USD.
    \todo{Can we somehow define ``and it does this without overcorrections'' in
    this acceptance criterium?
    I.e. we want to make sure that no auctions get triggered which bring the
    price to the other side of the USD value.}
  \item Admin and only admin can set the protocol-wide parameters (stability
    fee, liquidation discount, liquidation fee, liquidation ratio and debt
    floor)
  \item If the protocol-wide parameters are changed, this change is immediately
    and correctly seen in liquidations and when taking out loans
  % Comment for Version 2: Add debt ceiling to the protocol-wide parameters
\end{itemize}

\subsection{Update mechanism for the protocol (TBD)}

\section{Implementation details}

This section discusses what we need in order to implement the interactions
mentioned above.

\subsection{Price module}

The price module is a UTXO which can only be created by admin.
It contains in its datum a map from timestamps (POSIX time) to the price value.
An off-chain bot will be written which aggregates price information from at
least five sources (CEXes and DEXes) and applies the appropriate calculations on
this information in order to conclude what the most accurate current price is.
This current price will then be submitted as a transaction to the price module
using one of the admin keys.

Off-chain bot:
\begin{itemize}
  \item Haskell script that collects information and submits a transaction
  \item Before collecting any price information, the script reads out the
    relevant UTXO's datum and checks if it has been updated in the last hour. If
    so, the script terminates successfully right away.
  \item The script is ran once every five minutes through a systemd service
\end{itemize}

Price module:
\begin{itemize}
  \item The on-chain code will be written in Plutarch
  \item The off-chain code will be built as a PAB
  \item The datum will contain a map from timestamps (in POSIX time) to price
    values
  \item The script address will be dependent on three admin public keys
    \todo{Will these admin keys be updatable? \\
    Proposal: No. Can only be updated when the protocol is updated.
    That should be often enough for all practical purposes, and avoids having to
    write more code as well as avoiding the possibility of certain
    vulnerabilities. This decision can be enforced by using the keys to
    determine the script address, rather than putting them into the datum.}
\end{itemize}

Note: The off-chain bot will combine the price feeds through two operations:
\begin{enumerate}
  \item For each price feed, average the price over the values from the last
    hour
  \item Take the median of the resulting values
\end{enumerate}

% \subsection{Price oracle}
% 
% Each price oracle will carry an NFT used as a unique identifier.
% In addition, each price oracle contains the public key associated with the
% wallet which created it, so that only the creator of the oracle can update it.
% The price oracle can be updated once every hour.
% We will start with $3-5$ price oracles. \\
% 
% The OSM can be updated once every hour, and this transaction can be created by
% anyone.
% When it does, it pulls in the price information from all accepted price oracles
% and applies two calculations to that feed.
% First, it only retains the $70\%$ of price oracles that have most recently been
% updated, to assure obtaining the most recent price available.
% Secondly, it takes the median of the resulting price values.
% 
% \todo{How will price oracle updates and the OSM updates be rewarded? With a
% slight fee every time, coming from the profits stored in the buffer? Or should
% using a price oracle or OSM cost money, also to dUSD protocol users?}
% 
% If the OSM has not been updated for at least two hours, it stops spreading price
% information, thereby shutting down vault liquidations as well as creating new
% loans until the OSM is updated.
% This is a measure which will only kick in in the case of very serious resource
% contention, in which case once the system starts up again, we want to ensure
% that updating the OSM is the first transaction that happens.
% 
% \todo{Should the price oracles and the OSM all be combined into one UTXO, or
% be split into separate UTXOs?}
% 
\subsection{Vault (TBC)}

\begin{itemize}
  \item On-chain code: Plutarch
  \item Off-chain code: PAB, tested through our own testing framework (Maeserat)
  \item Collecting information on one person's vaults as well as vaults in
    general, are implemented as off-chain endpoints within the PAB, to avoid
    having to build a separate backend for the dUSD system
  \item A vault is a UTXO which holds an NFT, as a unique ID, and the collateral
    assets
  \item The vault's address is dependent on the owner's public key. This ensures
    the key cannot be changed by any transactions.
  \item The vault's datum consists of two pieces of information: A timestamp of
    its last update and the amount of dUSD debt (including stability fee)
    calculated at that time
  \item Liquidation is only allowed if the current price and the price of an
    hour ago both state that the current collateralization ratio is below the
    liquidation ratio.
    This is a measure taken to allow vault owners a chance to make their vaults
    healthy whenever the price of the collateral dropped.
  \item All transactions must have a ``txInfoValidRange'' that's less than $12$
    hours long, to provide sufficiently accurate temporal information. This is
    enforced on-chain. The timestamp extracted from any transaction is the the
    beginning of this interval.
  \item When a dUSD loan gets paid back, the dUSD coins get burned in the vault.
    This means that the stability fee is burned as well, which drives up the
    dUSD price and allows for surplus auctions (rewarding governance owners).
    Note that the burning of the stability fees isn't directly linked to an
    equal amount of dUSD being minted in the buffer.
    Instead they are linked indirectly, through the former driving up the dUSD
    price and hence triggering surplus auctions.
\end{itemize}

\todo{How do we ensure nobody copies the protocol? \\
Proposal: We put an NFT in our buffer and price module when minting, and use
those NFTs' names as arguments to the vault's Plutus script. This keeps our
vaults separate from anyone starting a different protocol. \\
In addition, we can make it mandatory to put a certain number of DANA tokens
into the buffer at creation.}

Note: The frontend code will set the minimum collateralization ratio, which will
be adjustable by Ardana itself.

\subsection{Buffer (TBC)}

The buffer will contain in its datum a map from timestamps to the stability fee
set at that time.
We use a map to make the stability fee calculations as accurate as
possible.\footnote{
  Imagine you open a vault, take out a loan and then leave the vault alone for
  two years.
  Just before you pay back the loan, governance decides to increase the
  stability fee.
  We want to use the stability fee that was applicable to the two year period
  when calculating your stability fee wherever possible, rather than
  retro-actively increasing your debt.
}. \\

We put together this map by figuring out the time of execution of the
transaction that changes the stability fee, within that transaction itself.
This is done by setting the timestamp to be the beginning of the
``txInfoValidRange'' range\footnote{
  \url{https://playground.plutus.iohkdev.io/doc/haddock/plutus-ledger-api/html/Plutus-V1-Ledger-Contexts.html}},
where the on-chain validator enforces this range to be less than $12$ hours
long. \\

In order to prevent datum overflow\footnote{
  This means we want to prevent the datum from becoming too big}
we limit the size of the map to $20$ data points.
This means that any transaction that changes the stability fee, will both add a
new entry to the map, and remove the oldest entry in case there are more than
$20$ entries. \\

The buffer will be a UTXO that includes in its datum the following:
\begin{itemize}
  \item Map from timestamps to stability fees
  \item Liquidation discount
  \item Liquidation fee
  \item Liquidation ratio
  \item Debt floor
\end{itemize}
The buffer's script address will be dependent on the same three admin public
keys as the price module. \\

Note: However we implement the auctions, we might want to prevent them from
updating the buffer's UTXO, to avoid resource contention? \\

\todo{How should the surplus/debt ``auctions'' be implemented?}

\begin{enumerate}
  \item Classic auction: Highest bid in X blocks wins
  \item Use a DANA price module to determine the market exchange rate between
    dUSD and DANA, and put the dUSD/DANA to sell into a UTXO where anyone can
    buy it instantly at a $2\%$ discount compared to that market exchange rate
  \item Only allow the dUSD/DANA exchange to happen on one of a number of
    trusted DEXes, incentivizing the broker with $1\%$ of the profits
\end{enumerate}

The main advantage of the price modules option, is that we always sell at a
trusted price, without allowing as many opportunities for manipulations.
(Classic auctions are vulnerable for manipulations through resource contention,
DEXes through executing big transactions, e.g. sandwich attacks.)
The trusted DEXes would be the fastest solution.

\section{Tests}


% \section{Remarks}
% 
% - Test for Cardano network congestion
% - Test what happens if someone hacks one or more price oracles
% - Other attack vectors which are not specific to our application, should be
%   listed and tested for


\end{document}
